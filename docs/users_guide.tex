
\documentclass{report}
\usepackage[T1]{fontenc}
\usepackage{ae, aecompl}			% Required for MNRAS
\usepackage{newtxtext, newtxmath} 	% Required for MNRAS
\usepackage{mathtools}
\usepackage{graphicx}
\usepackage{amsmath}
% \usepackage{amssymb}
\usepackage{multicol}
\usepackage{bm}
\usepackage{pdflscape}
\usepackage{natbib}
% \usepackage[section]{placeins}
\usepackage{lipsum}
\usepackage{etoolbox}
\usepackage{hyperref}
\usepackage{tabularx}
\usepackage[margin = 1in]{geometry}
\hypersetup{
	colorlinks 			= true, 
	urlcolor 			= blue, 
	linkcolor 			= blue, 
	citecolor 			= blue
}

\pagestyle{myheadings}

% Journal control sequences defined here because this is a LaTeX report 
\newcommand{\apj}{ApJ}
\newcommand{\apjs}{ApJS}
\newcommand{\aj}{AJ}
\newcommand{\araa}{ARA\&A}
\newcommand{\mnras}{MNRAS}
\newcommand{\pasa}{PASA}
\newcommand{\docsdir}{\url{https://github.com/giganano/VICE/tree/
master/docs}}

\begin{document}
\hypertarget{top}{}
\begin{center}
\underline{\LARGE
	\textbf{\texttt{VICE}: \texttt{Versatile Integrator for Chemical 
	Evolution}}
}
\par\null\par
{\LARGE \textbf{User's Guide}}
\par\null\par
{\Large \textbf{Version 1.0.0}}
\par\null\par
{\Large
James W. Johnson \& David H. Weinberg 
} \par
\textit{The Ohio State University, Department of Astronomy, 140 W. 18th 
Ave., Columbus, OH, 43204}

\par\null\par\noindent
\textbf{\texttt{VICE} is open-source software released under the MIT license. 
We invite developers and researchers to use, modify, and redistribute however 
they see fit under the terms of the associated license. \texttt{VICE}'s source 
code and installation instructions can be found 
at~\url{http://github.com/giganano/VICE.git}. Usage of \texttt{VICE} leading 
to a publication should cite Johnson \& Weinberg, 2019. }
\end{center}

\par\noindent
\underline{\textbf{Quick Links}} 
\par\noindent
\hyperlink{sec:cmdline}{\textbf{1) Command Line}}
\par\noindent
\hyperlink{sec:dataframes}{\textbf{2) The \texttt{VICE} Dataframe}}
\par
\hyperlink{df:atomic_number}{\texttt{vice.atomic\_number}}: 
\textit{Atomic number lookup for convenience} 
\par
\hyperlink{df:ccsne_yields}{\texttt{vice.ccsne\_yields}}:
\textit{Access, modify, and save yield settings from core collapse supernovae} 
\par
\hyperlink{df:sneia_yields}{\texttt{vice.sneia\_yields}}: 
\textit{Access, modify, and save yield settings from type Ia supernovae} 
\par
\hyperlink{df:solar_z}{\texttt{vice.solar\_z}}: 
\textit{Solar metallicity by mass lookup for convenience} 
\par
\hyperlink{df:sources}{\texttt{vice.sources}}: 
\textit{Dominant sources of enrichment lookup for convenience} 
\par\noindent
\hyperlink{sec:functions}{\textbf{3) Functions}}
\par
\hyperlink{func:agb_yield_grid}{\texttt{agb\_yield\_grid}}: 
\textit{Lookup mass-metallicity grid of fractional nucleosynthetic yields from 
AGB stars} 
\par
\hyperlink{func:fractional_cc_yield}{\texttt{fractional\_cc\_yield}}: 
\textit{Calculate IMF-integrated nucleosynthetic yields from core-collapse 
supernovae} 
\par
\hyperlink{func:fractional_ia_yield}{\texttt{fractional\_ia\_yield}}: 
\textit{Calculate IMF-integrated nucleosynthetic yields from type Ia 
supernovae} 
\par
\hyperlink{func:history}{\texttt{history}}: 
\textit{Read in the time-evolution of the ISM metallicity from an output} 
\par
\hyperlink{func:mdf}{\texttt{mdf}}: 
\textit{Read in the resulting stellar metallicity distribution function from 
an output} 
\par
\hyperlink{func:mirror}{\texttt{mirror}}: 
\textit{Obtain an integrator with the same parameters as that which produced 
a given output} 
\par
\hyperlink{func:single_ia_yield}{\texttt{single\_ia\_yield}}: 
\textit{Lookup the mass yield of a given element from a single instance of 
a type Ia supernova} 
\par
\hyperlink{func:single_stellar_population}{\texttt{single\_stellar\_population}}: 
\textit{Simulate mass enrichment of a given element from only one population 
of stars}
\par\noindent
\hyperlink{sec:classes}{\textbf{4) Classes}} 
\par
\hyperlink{obj:singlezone}{\texttt{singlezone}}: 
\textit{Run simulations of GCE models}\par
\hyperlink{obj:output}{\texttt{output}}: 
\textit{Reads in the output from instances of the }\texttt{vice.singlezone}
\textit{ class}
\par
\hyperlink{obg:dataframe}{\texttt{dataframe}}: 
\textit{A data-storing object with case-insensitive lookup functionality}
\par\null\par

\noindent
\hypertarget{sec:cmdline}{\textbf{1) Command Line}} \par\noindent 
We include a command-line entry with \texttt{VICE}, allowing users to run 
mathematically simple evolutionary models directly from the linux shell. While 
the command-line capabilities of \texttt{VICE} are useful for their ease, 
\texttt{VICE} is severely limited when ran in this manner in comparison to its 
capabilities when opened in a \texttt{python} interpreter. In this environment, 
users are limited to their default nucleosynthetic yield settings and smooth 
evolutionary models. In \texttt{python}, \texttt{VICE} allows users to 
construct arbitrary functions of time to describe many evolutionary parameters, 
allowing for the simulation of chemical enrichment under much more complex 
parameter spaces. 
\par\null\par

% vice.dataframe 
\newpage
\null\par\noindent 
\begin{center}
\hypertarget{sec:dataframes}{
	\underline{\LARGE
		\textbf{2) The \texttt{VICE dataframe}}
	}
} 
\end{center}
\par\noindent
\texttt{VICE} provides a \texttt{dataframe} which can be indexed either by a 
column label (i.e. with a \texttt{python} variable of type \texttt{str}) or 
by row number (i.e. with a \texttt{python} variable of type \texttt{int}). 
For ease of use, indexing a \texttt{VICE} dataframe by column label is 
case-insensitive by construction. 
\par
Users can turn any \texttt{python} dictionary into a \texttt{VICE dataframe} 
by passing it directly to \texttt{vice.dataframe}. There are several of these 
structures that we include with \texttt{VICE} which are designed to behave 
in a specific manner for various reasons specific to each case. We refer users 
to the documentation of each individual \texttt{dataframe} for further details 
on how they vary. 

\par\null\par\noindent
All \texttt{VICE} dataframes have the following functions: 

\null\par\noindent % vice.dataframe.keys
\hypertarget{df:keys}{\textbf{2.1) \texttt{vice.dataframe.keys()}}}
\par\noindent 
Returns a list of the \texttt{dataframe} keys in their lower-case format. This 
is analagous to the \texttt{keys} function for \texttt{python} dictionaries. 
\par\null\par\noindent 
\underline{\textbf{Example Code}} \par\noindent 
\texttt{$>>>$ import vice} \par\noindent 
\texttt{$>>>$ example = vice.dataframe(\{``oNe'': 1, ``TwO'': 2\})} 
\par\noindent 
\texttt{$>>>$ example.keys()} \par\noindent 
\texttt{[``two'', ``one'']} \par\noindent 

\null\par\noindent % vice.dataframe.todict
\hypertarget{df:todict}{\textbf{2.2) \texttt{vice.dataframe.todict()}}} 
Returns a \texttt{python} dictionary analog of the dataframe. The keys to the 
dataframe will be entirely lower-case strings. 
\par\null\par\noindent 
\underline{\textbf{Example Code}} \par\noindent 
\texttt{$>>>$ import vice} \par\noindent 
\texttt{$>>>$ example = vice.dataframe(\{``oNe'': 1, ``TwO'': 2\})} 
\par\noindent 
\texttt{$>>>$ example.todict()} \par\noindent 
\texttt{\{``one'': 1, ``two'': 2\}} \par\noindent 

\vfill
\hyperlink{top}{Back to top}
\clearpage 

% vice.atomic_number 
\newpage
\noindent 
\begin{center}
\hypertarget{df:atomic_number}{
	\underline{\LARGE
		\textbf{2.3) \texttt{vice.atomic\_number}} 
	}
}
\end{center}
\par\noindent 
A \texttt{VICE dataframe} containing the number of protons in the nucleus of 
each of \texttt{VICE}'s recognized elements. By design, this \texttt{dataframe} 
does not support item assignment. 
\par\null\par\noindent 
\underline{\textbf{Example Code}} 
\par\noindent 
\texttt{$>>>$ import vice} \par\noindent 
\texttt{$>>>$ vice.atomic\_number[``fe'']} \par\noindent 
\texttt{26} \par\noindent 
\texttt{$>>>$ vice.atomic\_number[``ni'']} \par\noindent 
\texttt{28} \par\noindent 
\texttt{$>>>$ vice.atomic\_number[``au'']} \par\noindent 
\texttt{79} \par\noindent 

\vfill
\hyperlink{top}{Back to top}
\clearpage 

% vice.ccsne_yields 
\newpage
\noindent 
\begin{center}
\hypertarget{df:ccsne_yields}{
	\underline{\LARGE
		\textbf{2.4) \texttt{vice.ccsne\_yields}}
	}
}
\end{center}
A \texttt{VICE dataframe} containing the current yield settings from core 
collapse supernovae. This \texttt{dataframe} is customizable and allows users 
to pass callable \texttt{python} functions, which will be interpreted as 
functions of metallicity $Z$. At the time an instance of the 
\texttt{singlezone} class is ran, the nucleosynthetic yield settings adopted 
by the simulation for each element are pulled from here. 
\par\null\par\noindent 
\underline{\textbf{Example Code}} 
\par\noindent 
\texttt{$>>>$ import vice} \par\noindent 
\texttt{$>>>$ vice.ccsne\_yields[``n'']} \par\noindent 
\texttt{0.000578} \par\noindent 
\texttt{$>>>$ vice.ccsne\_yields[``n'']} \par\noindent 
\texttt{$>>>$ vice.ccsne\_yields[``n''] - lambda z: 0.005 * (z / 0.014)} 
\par\noindent 
\texttt{$>>>$ vice.ccsne\_yields[``n'']} \par\noindent 
\texttt{<function \_\_main\_\_.<lambda$>>$} \par\noindent 

\null\par\noindent % vice.ccsne_yields.factory_defaults 
\hypertarget{df:ccsne_yields:factory_defaults}{
	\textbf{2.4.1) \texttt{vice.ccsne\_yields.factory\_defaults()}}
}
\par\noindent
Revert the current nucleosynthetic yield settings for core collapse supernovae 
to their original defaults (when \texttt{VICE} was first installed). This will 
not save these settings as the new defaults; that can be achieved by calling 
\texttt{vice.ccsne\_yields.save\_defaults()} immediately following this 
function. 

\null\par\noindent % vice.ccsne_yields.restore_defaults 
\hypertarget{df:ccsne_yields:restore_defaults}{
	\textbf{2.4.2) \texttt{vice.ccsne\_yields.restore\_defaults()}}
}
\par\noindent
Revert the current nucleosynthetic yield settings for core collapse supernovae 
to their current defaults (which may not be the original defaults). This will 
not save these settings as the new defaults; that can be achieved by calling 
\texttt{vice.ccsne\_yields.save\_defaults} immediately following this 
function. 

\null\par\noindent % vice.ccsne_yields.save_defaults 
\hypertarget{df:ccsne_yields:save_defaults}{
	\textbf{2.4.3) \texttt{vice.ccsne\_yields.save\_defaults()}}
}
\par\noindent 
Save the current nucleosynthetic yield settings for core collapse supernovae 
as defaults. Regardless of future changes in the user's current \texttt{python} 
interpreter, calling this function will make it so that the current settings 
are what \texttt{VICE} adopts upon import. 

\vfill
\hyperlink{top}{Back to top}
\clearpage 

% vice.sneia_yields
\newpage
\noindent 
\begin{center}
\hypertarget{df:sneia_yields}{
	\underline{\LARGE
		\textbf{2.5) \texttt{vice.sneia\_yields}}
	}
}
\end{center}
\hypertarget{df:sneia_yields}{\textbf{2.5) \texttt{vice.sneia\_yields}}} 
\par\noindent 
A \texttt{VICE dataframe} containing the current nucleosynthetic yield settings 
for type Ia supernovae. This \texttt{dataframe} is customizable but does not 
allow users to pass callable functions. 
\par\null\par\noindent 
\underline{\textbf{Example Code}} 
\par\qquad 
\texttt{$>>>$ import vice} \par\noindent 
\texttt{$>>>$ vice.sneia\_yields[``fe'']} \par\noindent 
\texttt{0.00258} \par\noindent 
\texttt{$>>>$ vice.sneia\_yields[``fe''] = 0.0017} \par\noindent 
\texttt{$>>>$ vice.sneia\_yields[``fe'']} \par\noindent 
\texttt{0.0017} \par\noindent 
\texttt{$>>>$ vice.sneia\_yields[``fe''] = lambda z: 0.0017 * (z / 0.014)} 
\par\noindent 
\texttt{TypeError: This dataframe does not support functional attributes.} 
\par\noindent 

\null\par\noindent % vice.sneia_yields.factory_defaults 
\hypertarget{df:sneia_yields:factory_defaults}{
	\textbf{2.5.1) \texttt{vice.sneia\_yields.factory\_defaults()}}
}
\par\noindent 
Revert the current nucleosynthetic yield settings for type Ia supernovae to 
their original defaults (when \texttt{VICE} was first installed). This will 
not save these settings as the new defaults; that can be achieved by calling 
\texttt{vice.sneia\_yields.save\_defaults} immediately following this function. 

\null\par\noindent % vice.sneia_yields.restore_defaults 
\hypertarget{df:sneia_yields:restore_defaults}{
	\textbf{2.5.2) \texttt{vice.sneia\_yields.restore\_defaults()}}
}
\par\noindent 
Revert the current nucleosynthetic yield settings for type Ia supernovae to 
their current defaults (which may not be the original defaults). This will 
not save these settings as the new defaults; that can be achieved by calling 
\texttt{vice.sneia\_yields.save\_defaults} immediately following this function. 

\null\par\noindent % vice.sneia_yields.save_defaults 
\hypertarget{df:sneia_yields:save_defaults}{
	\textbf{2.5.3) \texttt{vice.sneia\_yields.save\_defaults()}}
}
\par\noindent 
Save the current nucleosynthetic yield settings from type Ia supernovae as 
defaults. Regardless of future changes in the user's current \texttt{python} 
interpreter, calling this functions will make it so that the current settings 
are what \texttt{VICE} adopts upon import. 

\vfill
\hyperlink{top}{Back to top}
\clearpage 

% vice.solar_z 
\newpage
\noindent 
\begin{center} 
\hypertarget{df:solar_z}{
	\underline{\LARGE
		\textbf{2.6) \texttt{vice.solar\_z}}
	}
}
\end{center}
\par\noindent 
A \texttt{VICE dataframe} containing the abundance by mass of elements in the 
sun. Solar abundances are derived from~\citet{Asplund2009}, and have been 
converted into a mass fraction via: 
\begin{equation}
Z_{\text{x},\odot} = \mu_\text{x}X_\odot 10^{(X/H) - 12}
\end{equation}
where $\mu_\text{x}$ is the mean molecular weight of the element in amu, 
$X_\odot$ is the solar hydrogen abundance by mass, and $(X/H) = \log_{10}(
N_\text{x}/N_\text{H}) + 12$, which is what~\citet{Asplund2009} reports. For 
these calculations we adopt $X_\odot$ = 0.73 also from~\citet{Asplund2009}. 
\par
This \texttt{dataframe} does not support user customization. 
\par\null\par\noindent 
\underline{\textbf{Example Code}} 
\par\noindent 
\texttt{$>>>$ import vice} \par\noindent 
\texttt{$>>>$ vice.solar\_z[``o'']} \par\noindent 
\texttt{0.00572} \par\noindent 
\texttt{$>>>$ vice.solar\_z[``fe'']} \par\noindent 
\texttt{0.00129} \par\noindent 
\texttt{$>>>$ vice.solar\_z[``fe''] = 0.0014} \par\noindent 
\texttt{TypeError: This dataframe does not support item assignment.} 
\par\noindent 

\vfill 
\hyperlink{top}{Back to top}
\clearpage 

% vice.sources 
\newpage
\noindent 
\begin{center}
\hypertarget{df:sources}{
	\underline{\LARGE
		\textbf{2.7) \texttt{vice.sources}} 
	}
}
\end{center}
A \texttt{VICE dataframe} containing strings denoting what astronomers 
generally believe to be the dominant enrichment channels for each element. 
These are included purely for convenience. The fields of this 
\texttt{dataframe} for each element are adopted from~\citet{Johnson2019}. 
\par\null\par\noindent 
\underline{\textbf{Example Code}} 
\par\noindent 
\texttt{$>>>$ import vice} \par\noindent 
\texttt{$>>>$ vice.sources[``o'']} \par\noindent  
\texttt{[``CCSNE'']} \par\noindent 
\texttt{$>>>$ vice.sources[``fe'']} \par\noindent 
\texttt{[``CCSNE'', ``SNEIA'']} \par\noindent 
\texttt{$>>>$ vice.sources[``sr'']} \par\noindent  
\texttt{[``CCSNE'', ``AGB'']} \par\noindent 
\texttt{$>>>$ vice.sources[``au'']} \par\noindent 
\texttt{[``AGB'', ``NSNS'']} \par\noindent 
\texttt{$>>>$ vice.sources[``au''] = [``CCSNE'', ``AGB'']} \par\noindent 
\texttt{TypeError: This dataframe does not support item assignment.} 
\par\noindent 

\vfill
\hyperlink{top}{Back to top}
\clearpage 

% vice.agb_yield_grid
\newpage
\noindent 
\begin{center}
\hypertarget{func:agb_yield_grid}{
	\underline{\LARGE
		\textbf{3.1) \texttt{vice.agb\_yield\_grid}} 
	}
}
\end{center}
\par\noindent 
Obtain the stellar mass-metallicity grid of fractional nucleosynthetic yields 
from asymptotic giant branch (AGB) stars. \texttt{VICE} includes yields 
from~\citet{Karakas2010} and~\citet{Cristallo2011}, allowing users the choice 
of which to adopt in their simulations. 

\par\null\par\noindent
\textbf{Signature}: \texttt{vice.agb\_yield\_grid(element, study = 
``cristallo11'')}

\null\par\noindent
\underline{\textbf{Args}}
\par
\begin{itemize}
	\item{ % vice.agb_yield_grid.element 
		\texttt{element} (Type: \texttt{str}) \par
		The element to obtain the yield grid for. 
	}

	\item{ % vice.agb_yield_grid.study
		\texttt{study} (Default: \texttt{``cristallo11''}; Type: \texttt{str}) 
		\par
		A string (case-insensitive) denoting which AGB yield study to pull the 
		yield table from. In its current version, only \texttt{``karakas10''} 
		and \texttt{``cristallo11''} are the recognized values, denoting 
		the~\citet{Karakas2010} and~\citet{Cristallo2011} studies. 
	}
\end{itemize}

\null\par\noindent
\underline{\textbf{Returns}} 
\par\noindent 
A 3-element python list 
\begin{itemize}
	\item{
		\texttt{returned[0]}: A 2-D python list of yields that should be 
		indexed via \texttt{arr[mass\_index][z\_index]} 
	}

	\item{
		\texttt{returned[1]}: A python list containing the masses in M$_\odot$ 
		that the yield grid is sampled on. 
	}

	\item{
		\texttt{returned[2]}: A python list containing the metallicities by 
		mass that the yield grid is sampled on. 
	}
\end{itemize}

\par\null\par\noindent 
\underline{\textbf{Example Code}} 
\par\noindent 
\texttt{$>>>$ import vice} \par\noindent 
\texttt{$>>>$ y, m, z = vice.agb\_yield\_grid(``sr'')} \par\noindent 
\texttt{$>>>$ m} \par\noindent 
\texttt{[1.3, 1.5, 2.0, 2.5, 3.0, 4.0, 5.0, 6.0]} \par\noindent 
\texttt{$>>>$ z} \par\noindent 
\texttt{[0.0001, 0.0003, 0.001, 0.002, 0.003, 0.006, 0.008, 0.01, 0.014, 0.02]} 
\par\noindent 
\texttt{$>>>$ \# the fractional yield from 1.3 Msun stars at Z = 0.001}
\par\noindent 
\texttt{$>>>$ y[0][2]} \par\noindent 
\texttt{2.32254e-09}

\vfill
\hyperlink{top}{Back to top}
\clearpage 

% vice.fractional_cc_yield 
\newpage
\noindent 
\begin{center}
\hypertarget{func:fractional_cc_yield}{
	\underline{\LARGE 
		\textbf{3.2) \texttt{vice.fractional\_cc\_yield}}
	}
} 
\end{center}
\par\noindent 
Calculate an IMF-integrated fractional nucleosynthetic yield of a given element 
from core-collapse supernovae. \texttt{VICE} has built-in functions which 
implement Gaussian quadrature to evaluate these integrals numerically. See 
\texttt{VICE}'s science documentation for mathematical details. 

\par\null\par\noindent 
\textbf{Signature}: \texttt{vice.fractional\_cc\_yield(element, study = 
``LC18'', MoverH = 0, rotation = 0, IMF = ``kroupa'', method = ``simpson'', 
lower = 0.08, upper = 100, tolerance = 1e-3, Nmin = 64, Nmax = 2e8)} 

\null\par\noindent 
\underline{\textbf{Args}}
\par\noindent 
\begin{itemize} 
	\item{
		\texttt{element} (Type: \texttt{str}) 
		\par
		The element to calculate the IMF-integrated fractional yield for. 
	}

	\item{
		\texttt{study} (Default: \texttt{``LC18''}; Type: \texttt{str}) 
		\par
		A keyword (case-insensitive) denoting which study to adopt the yield 
		from. 
		\par
		Keywords and their associated studies: 
		\subitem{\texttt{``LC18''}: Limongi \& Chieffi (2018), ApJS, 237, 13}
		\subitem{\texttt{``CL13''}: Chieffi \& Limongi (2013), ApJ, 764, 21}
		\subitem{\texttt{``CL04''}: Chieffi \& Limongi (2004), ApJ, 608, 405} 
		\subitem{\texttt{``WW95''}: Woosley \& Weaver (1995), ApJ, 101, 181}
	}

	\item{
		\texttt{MoverH} (Default: 0; Type: real number) 
		\par
		The total metallicity [M/H] of the exploding stars. There are only a 
		handful of metallicities recognized by each study, and \texttt{VICE} 
		will raise a \texttt{ValueError} if this value is not one of them. 
		\par
		Keywords and their associated metallicities: 
		\subitem{\texttt{``LC18''}: [M/H] = -3, -2, -1, 0} 
		\subitem{\texttt{``CL13''}: [M/H] = 0} 
		\subitem{\texttt{``CL04''}: [M/H] = -inf, -4, -2, -1, -0.37, 0.15} 
		\subitem{\texttt{``WW95''}: [M/H] = -inf, -4, -2, -1, 0}
	}

	\item{
		\texttt{rotation} (Default: 0; Type: real number) 
		\par
		The rotational velocity $v_\text{rot}$ of the exploding stars in km/s. 
		There are only a handful of rotational velocities recognized by each 
		study, and \texttt{VICE} will raise a \texttt{ValueError} if this 
		value is not one of them. 
		\par 
		Keywords and their associated rotational velocities in km/s: 
		\subitem{\texttt{``LC18''}: $v_\text{rot}$ = 0, 150, 300} 
		\subitem{\texttt{``CL13''}: $v_\text{rot}$ = 0, 300} 
		\subitem{\texttt{``CL04''}: $v_\text{rot}$ = 0} 
		\subitem{\texttt{``WW95''}: $v_\text{rot}$ = 0} 
	}

	\item{
		\texttt{IMF} (Default: \texttt{``kroupa''}; Type: \texttt{str}) 
		\par
		The stellar initial mass function (IMF) to adopt. This must be either 
		\texttt{``kroupa''} for the~\citet{Kroupa2001} IMF or 
		\texttt{``salpeter''} for the~\citet{Salpeter1955} IMF 
		(case-insensitive). 
	}

	\item{
		\texttt{method} (Default: \texttt{``simpson''}; Type: \texttt{str}) 
		\par 
		The method of quadrature. See section XX.XX of \texttt{VICE}'s 
		science documentation for details.  
		\par
		Recognized methods: 
		\subitem{\texttt{``simpson''}}
		\subitem{\texttt{``trapezoid''}}
		\subitem{\texttt{``midpoint''}}
		\subitem{\texttt{``euler''}}
	}

	\item{
		\texttt{lower} (Default: 0.08; Type: real number) 
		\par
		The lower mass limit on star formation in solar masses. 
	}

	\item{
		\texttt{upper} (Default: 100; Type: real number) 
		\par
		The upper mass limit on star formation. 
	}

	\item{
		\texttt{tolerance} (Default: $10^{-3}$; Type: real number) 
		\par
		The numerical tolerance. The subroutines implementing Gaussian 
		quadrature in \texttt{VICE} will not return a result until the 
		fractional change between two successive integrations is smaller than 
		this value. 
	}

	\item{
		\texttt{Nmin} (Default: 64; Type: integer) 
		\par
		The minimum number of bins in quadrature. 
	}

	\item{
		\texttt{Nmax} (Default: $2\times10^8$; Type: integer) 
		\par
		The maximum number of bins in quadrature. Included as a failsafe 
		against non-convergent solutions. 
	}
\end{itemize}

\null\par\noindent 
\underline{\textbf{Returns}} 
\par\noindent 
A 2-element \texttt{python} list. 
\par\noindent
\begin{itemize} 
	\item{
		\texttt{returned[0]}: The numerically determined solution, with an 
		estimated precision below the specified tolerance provided the 
		solution converges within the maximum allowed number of bins in 
		quadrature. 
	}

	\item{
		\texttt{returned[1]}: The estimated fractional error. 
	}
\end{itemize}

\par\null\par\noindent 
\underline{\textbf{Example Code}} 
\par\noindent 
\texttt{$>>>$ import vice} \par\noindent 
\texttt{$>>>$ y, err = vice.fractional\_cc\_yield(``o'')} \par\noindent 
\texttt{$>>>$ y} \par\noindent 
\texttt{0.005643252355030168} \par\noindent 
\texttt{$>>>$ err} \par\noindent 
\texttt{4.137197161389483e-06} \par\noindent 
\texttt{$>>>$ y, err = vice.fractional\_cc\_yield(``mg'', study = ``CL13'')} 
\par\noindent 
\texttt{$>>>$ y} \par\noindent 
\texttt{0.000496663271667762} \par\noindent 

\vfill 
\hyperlink{top}{Back to top}
\clearpage 

% vice.fractional_ia_yield
\newpage 
\noindent 
\begin{center}
\hypertarget{func:fractional_ia_yield}{
	\underline{\LARGE 
		\textbf{3.3) \texttt{vice.fractional\_ia\_yield}}
	}
}
\end{center}
\par\noindent 
Calculate an IMF-integrated fractional nucleosynthetic yield of a given element 
from type Ia supernovae. Unlike \texttt{vice.fractional\_cc\_yield}, this 
function does not require numerical quadrature. 

\par\null\par\noindent 
\textbf{Signature}: \texttt{vice.fractional\_ia\_yield(element, 
	study = ``seitenzahl13'', model = ``N1'', n = $2.2\times10^{-3}$)} 

\par\null\par\noindent 
\underline{\textbf{Args}} 
\begin{itemize} 
	\item{
		\texttt{element} (Type: \texttt{str}) 
		\par
		The symbol of the element to calculate the yield for (case-insensitive). 
	}

	\item{
		\texttt{study} (Default: \texttt{``seitenzahl13''}; Type: \texttt{str}) 
		\par 
		A keyword (case-insensitive) denoting which study to adopt the yield 
		from. 
		\par
		Keywords and their associated studies: 
		\subitem{\texttt{``seitenzahl13''}: Seitenzahl et al. (2013), MNRAS, 
			429, 1156} 
		\subitem{\texttt{``iwamoto99''}: Iwamoto et al. (1999), ApJ, 124, 439} 
	}

	\item{
		\texttt{model} (Default: \texttt{``N1''}; Type: \texttt{str}) 
		\par 
		The model from the associated study to adopt. 
		\par
		Keywords and their associated models: 
		\subitem{\texttt{``seitenzahl13''}: N1, N3, N5, N10, N40, N100H, N100, 
		N100L, N150, N200, N300C}
		\subitem{\texttt{``iwamoto99''}: W7, W70, WDD1, WDD2, WDD3, CDD1, CDD2} 
	}

	\item{
		\texttt{n} (Default: \texttt{$2.2\times10^{-3}$}; Type: real number) 
		\par 
		The average number of type Ia supernovae produced per unit stellar 
		mass formed N$_\text{Ia}$/M$_*$. This parameter has units 
		M$_\odot^{-1}$. We recommend the default value of $2.2\times10^{-3}$ 
		M$_\odot^{-1}$ in accordance with Maoz \& Mannucci (2012), PASA, 29, 
		447. 
	}
\end{itemize}

\par\null\par\noindent 
\underline{\textbf{Returns}} 
\par\noindent 
The IMF-integrated yield. Unlike \texttt{vice.fractional\_cc\_yield}, there is 
no associated numerical error with this function, because the solution is 
analytic. 

\par\null\par\noindent 
\underline{\textbf{Example Code}} 
\par\noindent 
\texttt{$>>>$ import vice} \par\noindent 
\texttt{$>>>$ vice.fractional\_ia\_yield(``fe'')} \par\noindent 
\texttt{0.0025825957080000002} \par\noindent 
\texttt{$>>>$ vice.fractional\_ia\_yield(``ca'')} \par\noindent 
\texttt{8.935489894764334e-06} \par\noindent 
\texttt{$>>>$ vice.fractional\_ia\_yield(``ni'')} \par\noindent 
\texttt{0.00016576890932800003} \par\noindent 

\vfill 
\hyperlink{top}{Back to top} 
\clearpage 

% vice.history
\newpage
\noindent 
\begin{center}
\hypertarget{func:history}{
	\underline{\LARGE 
		\textbf{3.4) \texttt{vice.history}}
	}
}
\end{center}
\par\noindent 
Read in the part of a simulation's output that records the time-evolution of 
the ISM metallicity. 
\par\noindent 
Note: For an output under a given name, this file will be stored at 
\textit{name}.vice/history.out, and it is a simple ascii text file with a 
comment header detailing each column. By storing the output in this manner, 
we allow user's to analyze the results of \texttt{VICE} simulations in 
languages other than \texttt{python}. 

\par\null\par\noindent 
\textbf{Signature}: \texttt{vice.history(name)} 

\par\null\par\noindent 
\underline{\textbf{Args}}
\begin{itemize} 
	\item{
		\texttt{name} (Type: \texttt{str}) 
		\par
		The name of the output to read the history from as a string, with or 
		without the `.vice' extension. 
	}
\end{itemize}

\par\null\par\noindent 
\underline{\textbf{Returns}} 
\par\noindent 
A \texttt{VICE} dataframe containing the time in Gyr, gas stellar masses in 
M$_\odot$, star formation and infall rates in M$_\odot$ yr$^{-1}$, inflow and 
outflow metallicities of each element, gas-phase mass and metallicities of 
each element, and every [X/Y] combination of abundance ratios for each output 
timestep. 

\par\null\par\noindent 
\underline{\textbf{Example Code}} 
\par\noindent 
\texttt{$>>>$ import vice} \par\noindent 
\texttt{$>>>$ hist = vice.history(``example'')} \par\noindent 
\texttt{$>>>$ hist.keys()} \par\noindent 
\texttt{[``z(fe)'',} \par\noindent 
\texttt{``mass(fe)'',} \par\noindent 
\texttt{``[o/fe]'',} \par\noindent 
\texttt{``z\_in(sr)'',} \par\noindent 
\texttt{``z\_in(fe)'',} \par\noindent 
\texttt{``z(sr)'',} \par\noindent 
\texttt{``[sr/fe]'',} \par\noindent 
\texttt{``z\_out(o)'',} \par\noindent 
\texttt{``mgas'',} \par\noindent 
\texttt{``mass(sr)'',} \par\noindent 
\texttt{``z\_out(sr)'',} \par\noindent 
\texttt{``time'',} \par\noindent 
\texttt{``sfr'',} \par\noindent 
\texttt{``z\_out(fe)'',} \par\noindent 
\texttt{``eta\_0'',} \par\noindent 
\texttt{``[o/sr]'',} \par\noindent 
\texttt{``z(o)'',} \par\noindent 
\texttt{``[o/h]'',} \par\noindent 
\texttt{``ifr'',} \par\noindent 
\texttt{``z\_in(o)'',} \par\noindent 
\texttt{``ofr'',} \par\noindent 
\texttt{``[sr/h]'',} \par\noindent 
\texttt{``[fe/h]'',} \par\noindent 
\texttt{``r\_eff'',} \par\noindent 
\texttt{``mass(o)'',} \par\noindent 
\texttt{``mstar'']} \par\noindent 
\texttt{$>>>$ print(``[O/Fe] at the end of the simulation: \%.2e'' \% 
(hist[``[o/fe]''][-1]))} \par\noindent 
\texttt{[O/Fe] at the end of the simulation: -3.12e-01} \par\noindent 

\vfill
\hyperlink{top}{Back to top}
\clearpage 

% vice.mdf 
\newpage 
\noindent 
\begin{center} 
\hypertarget{func:mdf}{
	\underline{\LARGE
		\textbf{3.5) \texttt{vice.mdf}} 
	}
}
\end{center}
\par\null\par\noindent
Read in the normalized stellar metallicity distribution functions at the final 
timestep of the simulation. 
\par\null\par\noindent 
\textbf{Signature}: \texttt{vice.mdf(name)} 
\par\null\par\noindent 
\underline{\textbf{Args}}
\begin{itemize} 
	\item{
		\texttt{name} (Type: \texttt{str})
		\par
		The name of the simulation output to read from, with or without the 
		`.vice' extension. 
	}
\end{itemize}
\par\null\par\noindent 
\underline{\textbf{Returns}} 
\par\noindent
A \texttt{VICE dataframe} containing bin edges and the value of the normalized 
stellar metallicity distribution in each [X/H] abundance and [X/Y] abundance 
ratio. These distributions are normalized such that the integral of the 
distribution over all bins is equal to 1, meaning that the recorded values 
should be interpreted as probability densities. 
\par\null\par\noindent 
\underline{\textbf{Example Code}} 
\par\noindent 
\texttt{$>>>$ import vice} \par\noindent 
\texttt{$>>>$ m = vice.mdf(``example'')} \par\noindent 
\texttt{$>>>$ m.keys()} \par\noindent 
\texttt{[``dn/d[sr/h],'',} \par\noindent 
\texttt{``dn/d[sr/fe],''} \par\noindent 
\texttt{``bin\_edge\_left,''} \par\noindent 
\texttt{``dn/d[o/h],''} \par\noindent 
\texttt{``dn/d[o/fe],''} \par\noindent 
\texttt{``dn/d[fe/h],''} \par\noindent 
\texttt{``bin\_edge\_right,''} \par\noindent 
\texttt{``dn/d[o/sr]'']} \par\noindent 
\texttt{$>>>$ print(``dn/d[O/Fe] in 65th bin: \%.2e'' \% 
(m[``dn/d[o/fe]''][65]))} \par\noindent 
\texttt{dn/d[O/Fe] in 65th bin: 1.41e-01} \par\noindent 
\texttt{$>>>$ print(``65th bin edges: \%.2e, \%.2e'' \% 
(m[65][``bin\_edge\_left''], m[65][``bin\_edge\_right'']))} \par\noindent 
\texttt{65th bin edges: 2.50e-01, 3.00e-01} \par\noindent 

\vfill 
\hyperlink{top}{Back to top} 
\clearpage 

% vice.mirror 
\newpage 
\noindent
\begin{center}
\hypertarget{func:mirror}{
	\underline{\LARGE
		\textbf{3.6) \texttt{vice.mirror}}
	}
}
\end{center}
\par\null\par\noindent
Obtain an instance of the \texttt{vice.singlezone} class given only an 
instance of the \texttt{vice.output} class. The returned \texttt{singlezone} 
object will have the same parameters as that which produced the \texttt{output}, 
allowing re-simulation with whatever modifications the user desires. 
\par\null\par\noindent 
\textbf{Signature}: \texttt{vice.mirror(output\_obj)} 
\par\null\par\noindent
\underline{\textbf{Args}}
\par\noindent
\begin{itemize}
	\item{
		\texttt{output\_obj} (Type: \texttt{vice.output}) 
		\par
		Any \texttt{vice.output} object. 
	}
\end{itemize}
\par\null\par\noindent
\underline{\textbf{Returns}}
\par\noindent 
A \texttt{vice.singlezone} object with the same attributes as that which 
produced the given output. 

\par\null\par\noindent 
\underline{\textbf{Example Code}} 
\par\noindent 
\texttt{$>>>$ import vice} \par\noindent 
\texttt{$>>>$ out = vice.output(``example'')} \par\noindent 
\texttt{$>>>$ new = vice.mirror(out)} \par\noindent 
\texttt{$>>>$ new.settings()} \par\noindent 
\texttt{Current Settings:} \par\noindent 
\texttt{=================} \par\noindent 
\texttt{tau\_ia ---------------------------> 1.5} \par\noindent 
\texttt{recycling ------------------> continuous} \par\noindent 
\texttt{z\_solar ------------------------> 0.014} \par\noindent 
\texttt{enhancement ------------> 1.0} \par\noindent 
\texttt{agb\_model ------------------> cristallo11} \par\noindent 
\texttt{ria ------------------------------------> plaw} \par\noindent 
\texttt{delay ------------------------------> 0.15} \par\noindent 
\texttt{imf ------------------------------------> kroupa} \par\noindent 
\texttt{smoothing ------------------> 0.0} \par\noindent 
\texttt{schmidt\_index ------> 0.5} \par\noindent 
\texttt{eta ------------------------------------> 2.5} \par\noindent 
\texttt{zin ------------------------------------> 0.0} \par\noindent 
\texttt{schmidt ------------------------> False} \par\noindent 
\texttt{elements ---------------------> (u'fe', u'sr', u'o')} \par\noindent 
\texttt{MgSchmidt ------------------> 6000000000.0} \par\noindent 
\texttt{func ---------------------------------> <function \_DEFAULT\_FUNC 
at 0x1109e06e0>} \par\noindent 
\texttt{dt ---------------------------------------> 0.01} \par\noindent 
\texttt{tau\_star ---------------------> 2.0} \par\noindent 
\texttt{name ---------------------------------> onezonemodel} \par\noindent 
\texttt{m\_lower ------------------------> 0.08} \par\noindent 
\texttt{m\_upper ------------------------> 100.0} \par\noindent 
\texttt{Mg0 ------------------------------------> 6000000000.0} \par\noindent 
\texttt{mode ---------------------------------> ifr} \par\noindent 
\texttt{bins ---------------------------------> [-3, -2.95, -2.9, ... , 0.9, 
0.95, 1]} \par\noindent 
\texttt{$>>>$ \# this reruns the simulation} \par\noindent 
\texttt{$>>>$ import numpy as np} \par\noindent 
\texttt{$>>>$ new.run(np.linspace(0, 10, 1001))} \par\noindent 

\vfill 
\hyperlink{top}{Back to top} 
\clearpage 

% vice.single_ia_yield 
\newpage 
\noindent
\begin{center}
\hypertarget{func:single_ia_yield}{
	\underline{\LARGE
		\textbf{3.7) \texttt{vice.single\_ia\_yield}}
	}
}
\end{center}
\par\noindent 
Lookup the mass yield of a given element from a single instance of a type Ia 
supernova as determined by a given study and explosion model. 
\par\null\par\noindent 
\textbf{Signature}: \texttt{vice.single\_ia\_yield(element, 
study = ``seitenzahl13'', model = ``W7'')}
\par\null\par\noindent
\underline{\textbf{Args}}
\begin{itemize}
	\item{
		\texttt{element} (Type: \texttt{str})
		\par
		The element to look up the yield for. 
	}

	\item{
		\texttt{study} (Default: \texttt{``seitenzahl13''}; Type: \texttt{str}) 
		\par
		A keyword (case-insensitive) denoting which study to adopt the yield 
		from. 
		\par
		Keywords and their associated studies: 
		\par\qquad 
		\texttt{``seitenzahl13''}: Seitenzahl et al. (2013), MNRAS, 429, 1156 
		\par\qquad 
		\texttt{``iwamoto99''}: Iwamoto et al. (1999), ApJ, 124, 439
	}

	\item{
		\texttt{model} (Default: \texttt{``N1''}; Type: \texttt{str}) 
		\par 
		The model from the associated study to adopt (case-insensitive). 
		\par
		Keywords and their associated models: 
		\par\qquad 
		\texttt{``seitenzahl13''}: N1, N3, N5, N10, N40, N100H, N100, N100L, 
		N150, N200, N300C 
		\par\qquad 
		\texttt{``iwamoto99''}: W7, W70, WDD1, WDD2, WDD3, CDD1, CDD2 
	}
\end{itemize}
\par\null\par\noindent 
\underline{\textbf{Returns}} 
\par\noindent 
The mass yield of the given element in solar masses as determined for the 
specified model from the specified study. 

\par\null\par\noindent 
\underline{\textbf{Example Code}} 
\par\noindent 
\texttt{$>>>$ import vice} \par\noindent 
\texttt{$>>>$ vice.single\_ia\_yield(``fe'')} \par\noindent 
\texttt{1.17390714} \par\noindent 
\texttt{$>>>$ vice.single\_ia\_yield(``fe'', study 
model = ``W70'')} \par\noindent 
\texttt{0.77516} \par\noindent 
\texttt{$>>>$ vice.single\_ia\_yield(``ni'', model = ``N100L'')} \par\noindent 
\texttt{0.0391409000000526} \par\noindent 

\vfill 
\hyperlink{top}{Back to top}
\clearpage 

% vice.single_stellar_population
\newpage 
\noindent 
\begin{center} 
\hypertarget{func:single_stellar_population}{
	\underline{\LARGE
		\textbf{3.8) \texttt{vice.single\_stellar\_population}}
	}
}
\end{center}
\par\null\par\noindent 
Simulate the nucleosynthesis of a given element from a single star cluster of 
given mass and metallicity. This does not take into account galactic evolution 
- whether or not it is depleted from inflows or ejected in winds is not 
considered. Only the mass of the given element produced by the star cluster 
is determined. 
\par\null\par\noindent 
\textbf{Signature}: \texttt{vice.single\_stellar\_population} 
\par\noindent 
\begin{itemize} 
	\item{
		\texttt{element} (Type: \texttt{str}) 
		\par 
		The symbol of the element to simulate the enrichment for 
		(case-insensitive) 
	}

	\item{
		\texttt{mstar} (Default: $10^6$ M$_\odot$; Type: real number) 
		\par
		The total mass of the star cluster in solar masses when it is born 
		(at time = 0). 
	}

	\item{
		\texttt{Z} (Default: 0.014; Type: real number) 
		\par
		The metallicity by mass (i.e. the mass fraction of elements heavier 
		than helium) of the stars in the cluster. 
	}

	\item{
		\texttt{time} (Default: 10; Type: real number) 
		\par
		The amount of time in Gyr to run the simulation for. 
	}

	\item{
		\texttt{dt} (Default: 0.01; Type: real number) 
		\par
		The size of each timestep in Gyr. 
	}

	\item{ 
		\texttt{m\_upper} (Default: 100; Type: real number) 
		\par
		The upper mass limit on star formation in solar masses. 
	}

	\item{
		\texttt{m\_lower} (Default: 0.08; Type: real number) 
		\par
		The lower mass limit on star formation in solar masses. 
	}

	\item{
		\texttt{IMF} (Default: \texttt{``kroupa''}; Type: \texttt{str}) 
		\par
		The stellar initial mass function (IMF) to assume as a string 
		(case-insensitive). Currently only \texttt{``kroupa''} for the Kroupa 
		(2001), MNRAS, 322, 231 IMF and \texttt{``salpeter''} for the Salpeter 
		(1955), ApJ, 121, 161 IMF are recognized. 
	}

	\item{
		\texttt{RIa} (Default: \texttt{``plaw''}; Type: \texttt{str} or 
		callable function)
		\par
		The delay-time distribution $R_\text{Ia}(t)$ to adopt, where time is 
		in Gyr. \texttt{VICE} will automatically normalize any function that 
		is passed. Alternatively, \texttt{VICE} has built-in \texttt{``plaw''} 
		(power-law, $\propto t^{-1.1}$) and \texttt{``exp''} 
		(exponential, $\propto e^{-t /\text{1.5 Gyr}}$) delay-time 
		distributions. 
	}

	\item{
		\texttt{delay} (Default: 0.15; Type: real number) 
		\par
		The minimum delay time for the onset of type Ia supernovae in Gyr. 
	}

	\item{
		\texttt{agb\_model} (Default: \texttt{``cristallo11''}; Type: 
		\texttt{str}) 
		\par 
		A keyword (case-insensitive) denoting which table of nucleosynthetic 
		yields from AGB stars to adopt. 
		\par
		Recognized keywords and their associated studies: 
		\par\qquad 
		\texttt{``cristallo11''}: Cristallo et al. (2011), ApJS, 197, 17 
		\par\qquad 
		\texttt{``karakas10''}: Karakas (2010), MNRAS, 403, 1413 
	}
\end{itemize}

\par\null\par\noindent 
\underline{\textbf{Returns}} 
\par\noindent 
A 2-element \texttt{python list}. 
\begin{itemize} 
	\item{
		\texttt{returned[0]}: A \texttt{python list} containing the net mass 
		of the given element produced by the star cluster at each timestep. 
		Units are solar masses. 
	}

	\item{
		\texttt{returned[1]}: A \texttt{python list} containing the times in 
		Gyr corresponding to each mass yield. 
	}
\end{itemize} 

\par\null\par\noindent 
\underline{\textbf{Example Code}} 
\par\noindent 
\texttt{$>>>$ import vice} \par\noindent 
\texttt{$>>>$ mass, times = vice.single\_stellar\_population(``sr'', 
mstar = 1e6, Z = 0.008)} \par\noindent 
\texttt{$>>>$ \# Net strontium yield of a 1e6 Msun star cluster w/metallicity 
Z $=$ 0.008} \par\noindent 
\texttt{$>>>$ print(``\%.2e Msun'' \% (mass[-1]))} \par\noindent 
\texttt{4.81e-02 Msun} \par\noindent 
\texttt{$>>>$ mass, times = vice.single\_stellar\_population(``fe'', 
mstar = 1e6)} \par\noindent 
\texttt{$>>>$ \# Net iron yield of a 1e6 Msun star cluster w/metallicity 
Z $=$ 0.014} \par\noindent 
\texttt{$>>>$ print(``\%.2e Msun'' \% (mass[-1]))} \par\noindent 
\texttt{2.68e+03 Msun} \par\noindent  

\vfill
\hyperlink{top}{Back to top}
\clearpage 
% \texttt{$>>>$ print(``Net mass yield of a 10^6 Msun star cluster with 
% metallicity Z \= 0.008: '')} \par\qquad

% A breakdown if its features is shown in table~\ref{tab:prob1}. 

% \begin{table}[!h]
% \caption{The dataframes, functions, and classes included in \texttt{VICE}. }
% \begin{tabularx}{\textwidth}{l @{\extracolsep{\fill}} l l}
% \hline
% \hline
% Dataframes & Functions & Classes \\ 
% \hline
% \texttt{atomic\_number} & \texttt{agb\_yield\_grid} & \texttt{integrator} \\ 
% \texttt{ccsne\_yields} & \texttt{fractional\_cc\_yield} & \texttt{output} \\ 
% \texttt{sneia\_yields} & \texttt{fractional\_ia\_yield} & \texttt{dataframe} \\ 
% \texttt{solar\_z} & \texttt{history} & \null \\ 
% \texttt{sources} & \texttt{mdf} & \null \\ 
% \null & \texttt{mirror} & \null \\ 
% \null & \texttt{single\_ia\_yield} & \null \\ 
% \null & \texttt{single\_stellar\_population} & \null \\ 
% \hline
% \end{tabularx}
% \label{tab:prob1}
% \end{table}

\newpage
\null\par\noindent
\hypertarget{sec:classes}{\textbf{4) Classes}} \par\noindent

\null\par\noindent
\hypertarget{obj:integrator}{\textbf{\texttt{vice.integrator} (class)}} 
\par 
Runs simulations of chemical enrichment under the single-zone approximation for 
user specified parameters. The organization structure of this class is simple; 
every attribute encodes information on a given galaxy evolution parameter. The 
only function in this class that the user has access to is the \texttt{run} 
function, which runs the simulation over the current settings. 

\par\null\par
\textbf{Signature}: \texttt{vice.integrator(self, name = ``onezonemodel'', 
func = \_globals.\_DEFAULT\_FUNC, \newline mode = ``ifr'', elements = 
[``fe'', ``sr'', ``o''], imf = ``kroupa'', eta = 2.5, enhancement = 1, 
\newline Zin = 0, recycling = ``continuous'', bins = \_globals.\_DEFAULT\_FUNC, 
delay = 0.15, dtd = \newline``plaw'', Mg0 = 6.0e9, smoothing = 0.0, 
tau\_ia = 1.5, tau\_star = 2.0, dt = 0.01, schmidt = \newline False, 
schmidt\_index = 0.5, MgSchmidt = 6.0e9, m\_upper = 100, m\_lower = 0.08, 
Z\_solar = \newline 0.014, agb\_model = ``cristallo11'')}

\null\par\noindent
\underline{\textbf{Attributes}}
\par
\begin{itemize}
	\item{ % vice.integrator.name 
		\texttt{name} (Default: \texttt{``onezonemodel''}; Type: \texttt{str}) 
		\par
		The name of the integrator. The output will be stored in a directory 
		under this name with the extension \texttt{``.vice''}. This can also 
		be of the form \texttt{``/path/to/directory/name''} and the output 
		will be stored there. The user need not interact with any of the 
		output files; the \texttt{vice.output} object is designed to read in 
		all of the results automatically. 
		\par
		Outputs are handled in this format because it allows users to open 
		the files in languages other than \texttt{python}, as the simulation 
		results are recorded in pure \texttt{ascii} text. Forcing a 
		\texttt{``.vice''} extension on the name of the directory allows users 
		to run commands in a linux kernel over all outputs in a directory via 
		\texttt{<command> *.vice}. Within each \texttt{VICE} output are two 
		\texttt{``.out''} files; \texttt{``history.out''} contains the 
		time-evolution of the ISM metallicity and \texttt{``mdf.out''} 
		contains the normed stellar metallicity distribution function. The 
		three \texttt{``.config''} output files contain \texttt{python} 
		dictionaries stored in a \texttt{pickle} with the information to 
		reconstruct the yield settings and \texttt{integrator} settings from 
		the output. 
	}

	\item{ % vice.integrator.func 
		\texttt{func} (Default: \texttt{\_globals.\_DEFAULT\_FUNC}; 
		Type: \texttt{<function>}) 
		\par
		A callable \texttt{python} function of time which returns a numerical 
		value. It must only take one parameter, which \texttt{VICE} will 
		interpret as time in Gyr. The value returned by this function will 
		represent either the gas infall history $\dot{M}_\text{in}$ in 
		M$_\odot$ yr$^{-1}$, the star formation history in $\dot{M}_*$ in 
		M$_\odot$ yr$^{-1}$, or the gas supply in M$_\odot$ at that time. 
		\par
		The default function returns the value 9.1 at all times. \texttt{VICE} 
		defaults to infall mode, meaning that would represent an constant 
		infall rate of 9.1 M$_\odot$ yr$^{-1}$ if neither of these parameters 
		were changed. 
		\par
		See user's notes on \hyperlink{note:pyfuncs}{functional attributes} and  
		\hyperlink{note:delta_funcs}{numerical delta functions} in 
		\texttt{VICE}. 
	}

	\item{ % vice.integrator.mode 
		\texttt{mode} (Default: \texttt{``ifr''}; Type: \texttt{str} - either 
		\texttt{``ifr''}, \texttt{``sfr''}, or \texttt{``gas''} 
		[case-insensitive]) 
		\par
		The interpretation of the attribute \texttt{func}. If \texttt{``ifr''} 
		it represents the infall rate in M$_\odot$ yr$^{-1}$; if 
		\texttt{``sfr''} it represents the star formation history in 
		M$_\odot$ yr$^{-1}$; if \texttt{``gas''} it represents the gas supply 
		in M$_\odot$. The argument to \texttt{func} will always be interpreted 
		as time in Gyr. 
	}

	\item{ % vice.integrator.elements 
		\texttt{elements} (Default: [``fe'', ``sr'', ``o'']; Type: array-like 
		- elements of type \texttt{str} [case-insensitive]) 
		\par
		The symbols for the elements to track the enrichment for. The more 
		elements that are tracked, the more precisely calibrated is the total 
		ISM metallicity at each timestep for responding to metallicity 
		dependent nucleosynthetic yields, but the longer each simulation will 
		take. In its current version, \texttt{VICE} simulates enrichment via 
		core collapse supernovae, type Ia supernovae, and asymptotic giant 
		branch stars for all 76 elements between carbon (``c'') and bismuth 
		(``bi''). 
		\par
		\texttt{VICE} is compatible with the \texttt{NumPy array} and 
		\texttt{Pandas DataFrame}, but is not dependent on either package. 
	}

	\item{ % vice.integrator.imf 
		\texttt{imf} (Default: \texttt{``kroupa''}; Type: \texttt{str} - 
		either \texttt{``kroupa''} or \texttt{``salpeter''}) 
		\par
		The assumed stellar initial mass function (IMF). \texttt{VICE} 
		currently recognizes the~\citet{Kroupa2001} and~\citet{Salpeter1955} 
		IMFs. A future update will likely include a wider sample of IMFs. 
	}

	\item{ % vice.integrator.eta 
		\texttt{eta} (Default: \texttt{2.5}; Type: real number or 
		\texttt{<function>}) 
		\par 
		The mass loading factor $\eta = \dot{M}_\text{out}/\dot{M}_*$ (the 
		ratio of the outflow to the star formation rate). This can also be a 
		callable \texttt{python} function taking exactly one parameter, which 
		will be interpreted as time in Gyr. 
		\par
		See user's notes on \hyperlink{note:pyfuncs}{functional attributes} 
		and  \hyperlink{note:delta_funcs}{numerical delta functions} in 
		\texttt{VICE}. 
		\par
		If the user changes the smoothing timescale via the attribute 
		\texttt{smoothing}, the relationship between the outflow rate and the 
		star formation rate becomes more complicated. See \texttt{VICE}'s 
		science documentation at \docsdir~or this parameter's 
		\texttt{docstring} for mathematical and numerical details. 
	}

	\item{ % vice.integrator.enhancement 
		\texttt{enhancement} (Default: \texttt{1}; Type: real number or 
		\texttt{<function>}) 
		\par
		The ratio of the outflow to ISM metallicities. This can also be a 
		callable \texttt{python} function taking exactly one parameter, which 
		will be interpreted as time in Gyr. 
		\par
		See user's notes on \hyperlink{note:pyfuncs}{functional attributes} 
		and  \hyperlink{note:delta_funcs}{numerical delta functions} in 
		\texttt{VICE}. 
	}

	\item{ % vice.integrator.Zin 
		\texttt{Zin} (Default: \texttt{0}; Type: real number, 
		\texttt{<function>}, or \texttt{python} dictionary [case-insensitive]) 
		\par
		The metallicity of gas inflow. This can either be a number, which will 
		apply to all elements, a \texttt{<function>} of time in Gyr which will 
		also apply to all elements, or a \texttt{python} dictionary mapping 
		elemental symbols [\texttt{VICE} will respond case-insensitively] to 
		either real numbers or callable functions of time in Gyr. This allows 
		the user to construct arbitrary functions of time for each element, 
		allowing them to simulate the effects of infall metallicities in full 
		generality. 
		\par 
		See user's notes on \hyperlink{note:pyfuncs}{functional attributes} 
		and  \hyperlink{note:delta_funcs}{numerical delta functions} in 
		\texttt{VICE}. 
	}

	\item{ % vice.integrator.recycling 
		\texttt{recycling} (Default: \texttt{``continuous''}; Type: real 
		number or \texttt{str} - if \texttt{str}, it must be 
		\texttt{``continuous''} [case-insensitive])
		\par
		The cumulative return fraction $r$. This is the mass fraction of a 
		single stellar population returned to the ISM as gas at the birth 
		metallicity of the stars. By default, \texttt{VICE} implements 
		continuous recycling off of a treatment of the IMF weighted by the 
		mass of remnants as modeled by~\citet{Kalirai2008}. If the user 
		specifies a numerical value, it must be between 0 and 1. In this case 
		it will represent the instantaneous recycling parameter 
		$r_\text{inst}$ as in the analytic model of~\citet{Weinberg2017}. 
		\par
		See \texttt{VICE}'s science documentation at \docsdir~or this 
		parameter's docstring for mathematical and numerical deatils. 
	}

	\item{ % vice.integrator.bins 
		\texttt{bins} (Default: \texttt{\_globals.\_DEFAULT\_BINS}; Type: 
		array-like - elements are real numbers) 
		\par
		The bins in each [X/H] logarithmic abundance measurements and each 
		[X/Y] abundance ratio to sort the normed stellar metallicity 
		distribution function into. By default, \texttt{VICE} sorts everything 
		into 0.01-dex width bins between [X/H] and [X/Y] = -3 and +1. 
		\par
		\texttt{VICE} is compatible with the \texttt{NumPy array} and 
		\texttt{Pandas DataFrame}, but is not dependent on either package. 
	}

	\item{ % vice.integrator.delay 
		\texttt{delay} (Default: 0.15; Type: real number) 
		\par
		The minimum delay time in Gyr for the onset of type Ia supernovae 
		associated with each episode of star formation. 
	}

	\item{ % vice.integrator.dtd 
		\texttt{dtd} (Default: \texttt{``plaw''}; Type: \texttt{str} 
		[case-insensitive] or \texttt{<function>}) 
		\par 
		The delay-time distribution for type Ia supernovae to adopt. If type 
		\texttt{str}, \texttt{VICE} will use built-in delay-time distributions 
		(DTDs). These are \texttt{``exp''} for DTDs which decay exponentially 
		with e-folding timescale set by the attribute \texttt{tau\_ia} and 
		\texttt{``plaw''} for a power law with index 1.1 (i.e. $R_\text{Ia} 
		\propto t^{-1.1}$). 
		\par
		Alternatively the user may pass their own function of time as 
		$R_\text{Ia}(t)$. The user need not worry about normalizing their 
		custom DTD; \texttt{VICE} will take care of that automatically. 
		\par
		See user's notes on \hyperlink{note:pyfuncs}{functional attributes} 
		and  \hyperlink{note:delta_funcs}{numerical delta functions} in 
		\texttt{VICE}. 
	}

	\item{ % vice.integrator.Mg0 
		\texttt{Mg0} (Default: $6\times10^9$; Type: real number) 
		\par
		The mass of the ISM gas at time $t$ = 0 in M$_\odot$. This parameter 
		only matters when the \texttt{integrator} is in infall mode. In gas 
		mode, \texttt{func}$(0)$ specifies the initial gas supply, and in star 
		formation mode, it is $\texttt{func}(0) * \texttt{tau\_star}(0)$. 
	}

	\item{ % vice.integrator.smoothing 
		\texttt{smoothing} (Default: $0$; Type: real number) 
		\par 
		The smoothing time in Gyr to adopt. This is the timescale on which 
		the star formation rate is time-averaged before determining the 
		outflow rate via the mass loading parameter $\eta$ (attribute 
		\texttt{eta}). 
		\par
		See \texttt{VICE}'s science documentation at \docsdir~or this 
		parameter's docstring for mathematical and numerical deatils. 
	}

	\item{ % vice.integrator.tau_ia 
		\texttt{tau\_ia} (Default: 1.5; Type: real number) 
		\par
		The e-folding timescale in Gyr of an exponentially decaying delay-time 
		distribution for type Ia supernovae. This parameter only matters when 
		the attribute \texttt{dtd = ``exp''}. 
	}

	\item{ % vice.integrator.tau_star 
		\texttt{tau\_star} (Default: 2.0; Type: real number or 
		\texttt{<function>}) 
		\par
		The star formation efficiency (SFE) timescale denoting the gas supply 
		per unit star formation; $\tau_* = M_\text{g}/\dot{M}_*$ in Gyr. In 
		obserational journal articles, this is often referred to as the 
		``depletion time''. This parameter is how the gas supply and star 
		formation rate are determined off of one another at each timestep. 
		\par
		See user's notes on \hyperlink{note:pyfuncs}{functional attributes} 
		and  \hyperlink{note:delta_funcs}{numerical delta functions} in 
		\texttt{VICE}. 
	}

	\item{ % vice.integrator.dt 
		\texttt{dt} (Default: 0.01; Type: real number) 
		\par
		The timestep size in Gyr. For fine timesteps with a given ending time 
		in the simulation, this affects the total integration time with a 
		$\Delta t^{-2}$ dependence. 
	}

	\item{ % vice.integrator.schmidt 
		\texttt{schmidt} (Default: \texttt{False}; Type: boolean) 
		\par 
		A boolean describing whether or not to use an implementation of 
		gas-dependent star formation efficiency~\citep[i.e. the 
		Kennicutt-Schmidt Law;][]{Schmidt1959,Leroy2008}. At each timestep, 
		the user-specified \texttt{tau\_star}, normalization \texttt{MgSchmidt}, 
		and \texttt{schmidt\_index} determine the star formation efficiency at 
		that timestep via: 
		\begin{equation}
		\tau_*^{-1} = \texttt{tau\_star}(t)^{-1} \Big(\frac{M_\text{g}}
		{\texttt{MgSchmidt}}\Big)^\texttt{schmidt\_index}
		\end{equation}
	}

	\item{ % vice.integrator.schmidt_index 
		\texttt{schmidt\_index} (Default: 0.5; Type: real number) 
		\par
		The power law index on star formation efficiency driven by the 
		Kennicutt-Schmidt Law~\citep{Schmidt1959,Leroy2008}. See 
		\texttt{schmidt} documentation for further details, or \texttt{VICE}'s 
		science documentation at \docsdir~for mathematical and numerical 
		details. 
	}

	\item{ % vice.integrator.MgSchmidt 
		\texttt{MgSchmidt} (Default: $6\times10^9$; Type: real number) 
		\par
		The normalization of the gas supply when the star formation efficiency 
		is driven by the Kennicutt-Schmidt Law~\citep{Schmidt1959,Leroy2008}. 
		In pracitce, this quantity should be comparable to a typical gas 
		supply of the simulated galaxy so that the actual star formation 
		efficiency at a given timestep is near the user-specified value. See 
		documentation for the attribute \texttt{schmidt} for further details. 
	}

	\item{ % vice.integrator.m_upper 
		\texttt{m\_upper} (Default: 100; Type: real number) 
		\par
		The upper mass limit on star formation in M$_\odot$. 
	}

	\item{ % vice.integrator.m_lower 
		\texttt{m\_lower} (Default 0.08; Type: real number) 
		\par 
		The lower mass limit on star formation in M$_\odot$. 
	}

	\item{ % vice.integrator.Z_solar 
		\texttt{Z\_solar} (Default: 0.014; Type: real number) 
		\par
		The metallicity of the sun by mass. We recommend the default value of 
		0.014 based on the findings of~\citet{Asplund2009}. 
	}

	\item{ % vice.integrator.agb_model 
		\texttt{agb\_model} (Default: \texttt{``cristallo11''}; Type: 
		\texttt{str} - either \texttt{``cristallo11''} or 
		\texttt{``karakas10''}) 
		\par 
		A string denoting the keyword for the stellar mass-metallicity grid of 
		yields from asymptotic giant branch (AGB) stars to adopt. In its 
		current version, these are the only yield options for AGB stars 
		recognized by \texttt{VICE}. Future updates will include a wider 
		sample of built-in yield tables, construct their own yield tables, 
		and potentially specify functions of mass and metallicity for each 
		element. 
	}

\end{itemize}

\par\noindent
\underline{\textbf{Functions}} 
\par\null\par\noindent
\texttt{vice.integrator.run(output\_times, capture = False, overwrite = False)} 

\par 
Runs \texttt{VICE}'s built-in timestep integration routines over the 
parameters built into the attributes of this \texttt{class}. Whether or not 
the user sets \texttt{capture = True}, the output files will be produced and 
can be read into a \texttt{vice.output} object at any time. 

\begin{itemize}
	\item{ % vice.integrator.run.output_times 
		\texttt{output\_times} (Type: array-like, elements are real number) 
		\par 
		An array of times in Gyr at which the simulation should write output. 
		This need not be sorted in any order. 
	}

	\item{ % vice.integrator.run.capture 
		\texttt{capture} (Default: False; Type: Boolean) 
		\par
		If \texttt{True}, will return a \texttt{vice.output} object for the 
		results of the simulation. 
	}

	\item{ % vice.integrator.run.overwrite 
		\texttt{overwrite} (Default: False; Type: Boolean) 
		\par
		If \texttt{True}, will force overwrite any files under the same name 
		as the output. If not, this acts as a halting function because 
		\texttt{VICE} will stop to confirm that the output files are to be 
		overwritten. 
	}
\end{itemize}

\null\par\noindent
\hypertarget{obj:output}{\textbf{\texttt{vice.output} (class)}} 
\par 
Reads in the output from the \texttt{vice.integrator} class and allows the user 
to access it easily in a case-insensitive manner. The results are read in 
automatically upon instantiating an output object. 

\par\null\par
\textbf{Signature}: \texttt{vice.output(name)}

\par\null\par\noindent
\underline{\textbf{Attributes}} 
\begin{itemize}
	\item{ % vice.output.name
		\texttt{name} (Type: \texttt{str}) 
		\par 
		The name of the \texttt{``.vice''} directory containing the output of 
		an \texttt{``integrator''} object. The \texttt{``.vice''} extension 
		need not be specified with the name. Upon instantiating an output 
		object, the results will be read automatically. 
	}

	\item{ % vice.output.ccsne_yields 
		\texttt{ccsne\_yields} (Type: \texttt{VICE dataframe}) 
		\par 
		A case-insensitive dataframe containing the yield settings from 
		core-collapse supernovae at the time the simulation was ran. 
	}

	\item{ % vice.output.elements 
		\texttt{elements} (Type: \texttt{tuple}) 
		\par 
		A \texttt{python tuple} containing the symbols of the elements whose 
		enrichment was tracked by the simulation. 
	}

	\item{ % vice.output.history 
		\texttt{history} (Type: \texttt{VICE dataframe})
		\par 
		A case-insensitive dataframe containing the time-evolution of the 
		galaxy and its abundances, such as the star formation rate, gas mass, 
		and abundances. This can be keyed on by either column name or 
		line number. For example: 
		\par$\qquad$ 
		\texttt{$>>>$ import vice} 
		\par$\qquad$
		\texttt{$>>>$ out = vice.output(``onezonemodel'')}
		\par$\qquad$
		\texttt{$>>>$ sfr = out.history[``sfr'']} 
		\par$\qquad$
		\texttt{$>>>$ ofe = out.history[``[o/fE]'']}
		\par$\qquad$ 
		\texttt{$>>>$ hundredth\_line = out.history[100]}
		\par\noindent
		where we have made deliberate case errors in the fourth line to 
		demonstrate that the \texttt{VICE dataframe} is case insensitive. In 
		the case of the final line, \texttt{hundredth\_line} will also be a 
		\texttt{VICE dataframe}. 
	}

	\item{ % vice.output.mdf 
		\texttt{mdf} (Type: \texttt{VICE dataframe}) 
		\par
		A case-insensitive dataframe containing the normalized stellar 
		metallicity distribution function at the final timestep of the 
		simulation. This quantity represents the probability density that a 
		random star would have a given [X/H] abundance or [X/Y] abundance 
		ratio in the given bin. This \texttt{dataframe} also contains the 
		bin edges that the user built-into the \texttt{integrator} to run 
		the simulation. 
		\par
		\textbf{NOTE}: If any [X/H] abundances or [X/Y] abundance ratios 
		determined by \texttt{VICE} never pass within the user's specified 
		binspace, then the associated MDF will be \texttt{NaN} at all values. 
	}

	\item{ %vice.output.sneia_yields 
		\texttt{sneia\_yields} (Type: \texttt{VICE dataframe}) 
		\par
		A case-insensitive dataframe containing the yield settings from 
		type Ia supernovae at the time the simulation was ran. 
	}
\end{itemize}

\par\null\par\noindent
\underline{\textbf{Functions}} 
\par\null\par\noindent
\texttt{vice.output.show(key)}
\par
Show a plot of the given quantity referenced by the argument \texttt{key} 
[case-insensitive]. If this is a quantity stored in the 
\texttt{vice.output.history dataframe}, it will be plotted against time by 
default. Conversely, if it is stored in the \texttt{vice.output.mdf dataframe}, 
it will show the corresponding stellar metallicity distribution function. 
\par
Users can also specify an argument of the format \texttt{key1-key2} where 
\texttt{key1} and \texttt{key2} are elements of the 
\texttt{vice.output.history dataframe}. It will then plot \texttt{key1} against 
\texttt{key2} and show it to the user. For example: 
\par$\qquad$
\texttt{$>>>$ import vice} 
\par$\qquad$ 
\texttt{$>>>$ out = vice.output(``example'')} 
\par$\qquad$ 
\texttt{\# Will show the star formation rate against time in Gyr}
\par$\qquad$ 
\texttt{$>>>$ out.show(``sfr'')}
\par$\qquad$ 
\texttt{\# Will show the stellar MDF in [O/Fe]} 
\par$\qquad$ 
\texttt{$>>>$ out.show(``dn/d[o/fe]'')}
\par$\qquad$ 
\texttt{\# Will show the track in the [O/Fe]-[Fe/H] plane} 
\par$\qquad$ 
\texttt{$>>>$ out.show(``[O/Fe]-[Fe/H]'')}
\par
\textbf{NOTE}: This function is not intended to generate publication quality 
plots for users. It is included purely as a convenience function for users to 
be able to read in and immediately inspect the results of their simulations 
in a plot with only a few lines of code. 
\par
This is the only function included with \texttt{VICE} that is dependent on 
any derivative of \texttt{anaconda}, requiring \texttt{matplotlib} version 
$\geq$ 2. 

\null\par\noindent
\hypertarget{note:pyfuncs}{\underline{\textbf{User's Note on Functional 
Attributes in \texttt{VICE}}}} \par\noindent 
Functional attributes in \texttt{VICE} must be native \texttt{python} 
functions. Any byte-compiled function, such as \texttt{NumPy} mathemetical 
functions or any \texttt{python} code that has been ran through a 
\texttt{Cython} compiler, will produce a \texttt{TypeError}. If the user 
wishes to employ one of these functions, this can be achieved by simply 
wrapping them in a \texttt{python} function. For example, the following will 
produce a \texttt{TypeError}: 
\par
\texttt{$>>>$ import numpy as np} \par
\texttt{$>>>$ import vice} \par
\texttt{$>>>$ intgtr = vice.integrator()} \par
\texttt{$>>>$ intgtr.func = np.exp} \par\noindent
but the following will not: \par
\texttt{$>>>$ import numpy as np} \par
\texttt{$>>>$ import vice} \par
\texttt{$>>>$ intgtr = vice.integrator()} \par 
\texttt{$>>>$ def f(t): } \par
\texttt{$>>>\qquad$ return np.exp(t) } \par
\texttt{$>>>$ intgtr.func = f} \par\noindent
The same can be achieved with a \texttt{python lambda}.  

\null\par\noindent
\hypertarget{note:delta_funcs}{\underline{\textbf{User's Note on Numerical 
Delta Functions}}} \par\noindent
\texttt{VICE} is a timestep-style integrator, and therefore, numerical delta 
functions can be  achieved by letting a quantity take on some very high value 
for one timestep. If the user wishes to build a delta function into their 
model, they need to make sure that: 
\begin{enumerate}
\item
They let their delta function have an intrinsic finite width of at least one 
timestep. Otherwise, it is not guaranteed that the numerical integrator will 
find the delta function. 

\item
They have set their output times such that the integrator will write to the 
output file at the time of the delta function. If this is not ensured, the 
output will still show the behavior induced by the delta function, but not in 
the parameter which was meant to exhibit one. 
\end{enumerate}
When running an integrator, we recommend that users set it to write output 
at every timestep for a brief period following a numerical delta function in 
any parameter. This simply ensures that the output will reflect more detail 
following its onset. See \texttt{vice.integrator.run} docstring for details 



\bibliographystyle{mnras}
\bibliography{science_documentation}

\end{document}



% \texttt{VICE} stores data locally in a \texttt{dataframe} which is designed to 
% emulate the \texttt{Pandas} dataframe, but in a case-insensitive manner for 
% ease of use. These dataframes can be indexed with a \texttt{str}. In the case 
% of the \texttt{dataframes} included in the \texttt{vice.output} object, an 
% \texttt{int} can also be passed, and it will return a \texttt{dataframe} 
% containing the value of each column at that line of output. The following is 
% an example showing the 100$^\text{th}$ line of output from a simulation over 
% the default parameters of the \texttt{integrator} class: 
% \par\qquad 
% \texttt{$>>>$ import vice} 
% \par\qquad 
% \texttt{$>>>$ out = vice.output(``example'')} 
% \par\qquad 
% \texttt{$>>>$ out.history[100]}
% \par\qquad
% \texttt{vice.dataframe\{} \par\qquad 
% \texttt{$\qquad$[fe/h] --------------------> -0.6565517} \par$\qquad$ 
% \texttt{$\qquad$z(fe) ----------------------> 0.000284471} \par$\qquad$ 
% \texttt{$\qquad$mass(sr) --------------> 21.54993} \par$\qquad$ 
% \texttt{$\qquad$sfr ----------------------------> 2.770642} \par$\qquad$ 
% \texttt{$\qquad$z\_out(sr) -----------> 3.888978e-09} \par$\qquad$ 
% \texttt{$\qquad$time --------------------------> 1.0} \par$\qquad$ 
% \texttt{$\qquad$mass(fe) --------------> 1576335.0} \par$\qquad$ 
% \texttt{$\qquad$[o/fe] --------------------> 0.05612465} \par$\qquad$ 
% \texttt{$\qquad$z\_out(fe) -----------> 0.000284471} \par$\qquad$ 
% \texttt{$\qquad$z\_in(sr) -------------> 0.0} \par$\qquad$ 
% \texttt{$\qquad$[o/sr] --------------------> 0.4855158} \par$\qquad$ 
% \texttt{$\qquad$z(o) -------------------------> 0.001435387} \par$\qquad$ 
% \texttt{$\qquad$z\_in(fe) -------------> 0.0} \par$\qquad$ 
% \texttt{$\qquad$[o/h] ----------------------> -0.600427} \par$\qquad$ 
% \texttt{$\qquad$z(sr) ----------------------> 3.888978e-09} \par$\qquad$ 
% \texttt{$\qquad$eta\_0 ----------------------> 2.5} \par$\qquad$ 
% \texttt{$\qquad$ifr ----------------------------> 9.1} \par$\qquad$ 
% \texttt{$\qquad$[sr/h] --------------------> -1.085943} \par$\qquad$ 
% \texttt{$\qquad$z\_in(o) ----------------> 0.0} \par$\qquad$ 
% \texttt{$\qquad$ofr -----------------------------> 6.926605} \par$\qquad$ 
% \texttt{$\qquad$[sr/fe] ----------------> -0.4293912} \par$\qquad$ 
% \texttt{$\qquad$mgas -------------------------> 5541284000.0} \par$\qquad$ 
% \texttt{$\qquad$r\_eff ----------------------> 0.4156579} \par$\qquad$ 
% \texttt{$\qquad$z\_out(o) -------------> 0.001435387} \par$\qquad$ 
% \texttt{$\qquad$mass(o) ----------------> 7953888.0} \par$\qquad$ 
% \texttt{$\qquad$mstar -----------------------> 1785906000.0} \par$\qquad$ 
% \texttt{\}} \par\qquad 
% \texttt{$>>>$ out.history[100][``[FE/H]'']} \par\qquad 
% \texttt{-0.6565517} \par\qquad 
% \texttt{$>>>$ out.history[``[o/fe]''][100:110]} \par\qquad 
% \texttt{[0.05612465,} \par\qquad 
% \texttt{ 0.05332706,} \par\qquad 
% \texttt{ 0.05057033,} \par\qquad 
% \texttt{ 0.04785348,} \par\qquad 
% \texttt{ 0.04517558,} \par\qquad 
% \texttt{ 0.04253571,} \par\qquad 
% \texttt{ 0.03993299,} \par\qquad 
% \texttt{ 0.03736658,} \par\qquad 
% \texttt{ 0.03483565,} \par\qquad 
% \texttt{ 0.03233941]} \par\noindent
% Users are not permitted to make their own instances of the 
% \texttt{VICE dataframe}. If users wish to emulate this functionality in their 
% own work, we direct them to the source code, implemented in the 
% \texttt{\_case\_insensitive\_dataframe} class at 
% \texttt{vice/core/\_globals.py}. 



